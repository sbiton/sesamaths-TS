\documentclass[10pt,a4paper]{article}

\usepackage{amsmath}
\usepackage[utf8x]{inputenc}
\usepackage[french]{babel}
\usepackage{exostyle,graphicx,fancybox}


% \newcommand{\elim}[1]{#1}
% \newcommand{\notep}[1]{{{\hfill\color{red}\it #1}}}
% \newcommand{\exemp}[1]{{\color{blue}\it\underline{\it Exemple \thePROPno } : #1}}
% \newcommand{\exer}[1]{{\color{green}\it\underline{\it Exercice} : #1}}
% \newcommand{\corr}[1]{{\it \color{red} \underline{ Corrig\'e} : #1 }}
% \newcommand{\preuve}[1]{{\it\scriptsize \underline{Preuve} : \\ #1}}
% %%%%%%%%%%%%%%%%%%%%%%   avec lien
% \newcommand{\monlien}[2]{\hfill\hyperlink{#1}{#2}}
% \newcommand{\macible}[1]{\hfill\hypertarget{#1}{}}
% % %%%%%%%%%%%%%%%%%%% en vidant ces commandes, on enl\`eve les liens et les notes.
% \renewcommand{\elim}[1]{}

\begin{document}



%  \newtheorem{mytheorem}[theorem]{{Propri\'et\'e} }
%  \newtheorem{mydefinition}[theorem]{{D\'efinition } }


\begin{center} 
{\bf\Large Forme exponentielle des nombres complexes} 
\end{center}



\underline{Introduction  de la notation :}

 Soit $f$ la fonction définie sur $\R$ par : $f(\theta)=\cos(\theta)+i\sin(\theta)$.

$f(\theta)$ est un nombre complexe de module 1 et d'argument $\theta$.
\begin{itemize}

 \item On a montré dans un cours précédent que : $\arg(z_1\times z_2)=\arg(z_1)+\arg(z_2)$. Donc :
 $f(\theta+\theta')=f(\theta)\times f(\theta')$.
\item De plus : $f(0)=1$

\end{itemize}

Par analogie avec la fonction exponentielle, on convient de noter : \fbox{$\cos(\theta)+i\sin(\theta)=e^{i\theta}$}

\section{Forme exponentielle des nombres complexes}
%\subsection{Introduction- Notation $e^{i\theta}$}


\subsection{\'Ecriture exponentielle des complexes de module 1}

{\bf Définition :}

\begin{tabular}{p{.5cm}||p{17cm}}
& Tout nombre complexe de module 1 et d'argument $\theta$ peut s'écrire sous la forme : 
\fbox{$\cos(\theta)+i\sin(\theta)=e^{i\theta}$}
%Les nombres complexes $z$ tels que $|z|=1$ s'\'ecrivent sous la forme \fbox{$z=e^{i\theta}=e^{i\arg(z)}$}
\end{tabular}


\medskip

{\it Exemple 1 : 
\begin{enumerate}
\item Placer sur le cercle trigonométrique les points $M_i$ d'affixes  : $z_1=e^{i\frac{\pi}{2}}$;  $z_2=e^{i\pi}$;  $z_3=e^{i\frac{3\pi}{2}}$;  $z_4=e^{i2\pi}$; $z_5=e^{i\frac{2\pi}{3}}$.
 \item Donner l'\'ecriture alg\'ebrique des 5 complexes précédents.
\end{enumerate}}

\medskip

{\it Exemple 2 : \'Ecrire sous forme $e^{i\theta}$ les complexes suivants : $z'_1=i$; $z'_2=-i$; $z'_3=\dfrac{-1}{2}i+ \dfrac{\sqrt{3}}{2}$}



\subsection{Cas g\'en\'eral}

{\bf Définition - Propriété }

\begin{tabular}{p{.5cm}||p{17cm}}
&
 Tout complexe $z\neq 0$  s'\'ecrit sous la forme \fbox{$z=re^{i\theta}$} avec 
\fbox{$r=|z|$ et $\theta\equiv \arg(z) [2\pi]$}.

Cette \'ecriture est appel\'ee \og forme exponentielle du complexe $z$\fg
\end{tabular}

\medskip
\begin{tabular}{p{.5cm}||p{17cm}}
&
Réciproque : Si $z\in \C^*$ et $z=re^{i\theta}$ avec $r>0$ alors $r=|z|$ et $\theta=\arg(z)$
\end{tabular}

\medskip

{\it Exemple 3 :\begin{enumerate}
\item \'Ecrire sous forme exponentielle : $z_1=-2i$, $z_2=1-i$,$z_3=2-2i\sqrt{3}$
 \item \'Ecrire sous forme alg\'ebrique : $z'=4e^{i\frac{2\pi}{3}}$
\end{enumerate}
}

\section{Calculs avec la notation exponentielle}

\subsection{Propri\'et\'es alg\'ebriques}

{\bf Propriétés :}


\begin{tabular}{p{.5cm}||p{17cm}}
&
Pour tous nombres r\'eels $\theta, \theta'$ :

\begin{center}
\fbox{ 
\begin{tabular}[c]{ll}
$e^{i\theta}\times e^{i\theta'} =e^{i(\theta+\theta')}$ & $(e^{i\theta})^n=e^{in\theta},\ \ n\in \Z$ \\ 
$\di \frac{1}{e^{i\theta}}=e^{-i\theta}=\overline{e^{i\theta}} $& 
$\di \frac{e^{i\theta}}{e^{i\theta'}} =e^{i(\theta-\theta')}$
\end{tabular}}
\end{center}
\end{tabular}

\medskip

\underline {Remarque :} Ces propriétés sont admises. Elles résultent du fait que $|e^{i\theta}|=1 $ et des propriétés des arguments.

\medskip

{\it Exemple 4 : 

 Mettre sous forme exponentielle :
\begin{enumerate}
 \item $z_1=-\sqrt{3}+i$, $z_2=e^{-i\frac{\pi}{6}}z_1^2$, $z_2=\dfrac{2z_1}{5e^{-i\frac{\pi}{6}}}$ 
 \item D\'eterminer les entiers $n$ tels que $z_1^n$ est un nombre r\'eel.
\end{enumerate}}

\medskip

{\it Exemple 5 : Retrouver les formules d'addition ($\cos(a+b)$, $\cos(a-b)$, $\sin(a+b)$ et $\sin(a-b)$) et 
de duplication ($\cos(2a)$, $\sin(2a)$) en utilisant les nombres complexes
}


\subsection{Formule d'Euler}

{\bf Propriété}


\begin{tabular}{p{.5cm}||p{17cm}}
&
Pour tout $\theta \in \R$ : 
\fbox{$\di \cos(\theta)=\frac{e^{i\theta}+e^{-i\theta}}{2}$}
\ \ \ et \ \ \ \ \fbox{$\di \sin(\theta)=\frac{e^{i\theta}-e^{-i\theta}}{2i}$}
\end{tabular}

\medskip
{\it Exemple 6 : lin\'eariser $\cos^2(x)$ et $\sin^2(x)$.
 

\end{document}




\subsection{Application g\'eom\'etrique : \'equation param\'etrique complexe d'un cercle}

\frameprop{}{
\vspace{-.2cm}

\begin{enumerate}
 \item l'ensemble des points $M$ du plan tels que $z=z_{\Omega}+Re^{i\theta}$ constitue le cercle ${\cal C}(\Omega,R)$.
\item l'\'ecriture pr\'ec\'edente est appel\'ee "\'equation complexe`` du cercle ${\cal C}(\Omega,R)$
\end{enumerate}
}{
\notep{faire la preuve}

\exemp{
\begin{enumerate}
\item Donner la repr\'esentation param\'etrique complexe du cercle de centre $A(1+2i)$ et de rayon 2
\item Repr\'esenter les points $M(z)$ du plan tels que $z=3+2e^{i\theta}$
\end{enumerate}
}
\exer{ pas pour l'instant}
}


% \newcommand{\exempProp4}[0]{
% Mettre sous forme exponentielle :
% \begin{enumerate}
%  \item $z_1=-\sqrt{3}+i$, $z_2=e^{-i\frac{\pi}{6}}z_1^2$, $z_2=\dfrac{2z_1}{5e^{-i\frac{\pi}{6}}}$ 
%  \item D\'eterminer les entiers $n$ tels que $z_1^n$ est un nombre r\'eels.
% \end{enumerate}
% }




\notep{Faire les preuves}

\exer{2,3 feuille  }

}

\end{document}
